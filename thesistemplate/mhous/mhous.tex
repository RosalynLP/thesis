\chapter{Missing higher order uncertainties}

In this chapter we address the dominant source of theoretical uncertainty in current PDF fits: missing higher order uncertainties (MHOUs). In Sec.~\ref{sec:intro} we explain their origin, then in Sec.~\ref{sec:svn} we revise their standard method of estimation, through scale variation. We then show how to use this to construct a theory covariance matrix (Sec.~\ref{sec:prescrip}), and test the validity of this at NLO against the known NNLO result (Sec.~\ref{sec:valid}). Finally, we present the PDFs including MHOUs (Sec.~\ref{sec:pdfs}) and assess the impact on relevant phenomenology (Sec.~\ref{sec:mhoupheno}).

\section{Introduction}
\label{sec:intro}
PDF fits rely on the comparison of experimental data with theoretical predictions at the partonic level. These predictions are carried out in the framework of perturbation theory, where results are expressed as an expansion in the strong coupling constant, $\alpha_s$. The first non-zero contribution to the expansion is known as ``leading order" (LO), the next is ``next-to-leading order" (NLO), and so on (NNLO, N$^3$LO etc.). Because in the perturbative regime $\alpha_s$ is small (0.118 \cite{pdg}), corrections from higher orders are increasingly small. Predictions must be directly calculated at each order by considering all the possible contributing Feynman diagrams, and this becomes exponentially more complicated with increasing orders; the cutting edge of calculations is currently at the N$^3$LO level. PDFs are fitted using predictions truncated at a given order, with NNLO PDFs being the modern standard. 

\section{Scale variation}
\label{sec:svn}

\subsection{Renormalisation group invariance}

\subsection{Scale variation in partonic cross sections}
\subsubsection{Electroproduction}
\subsubsection{Hadroproduction}

\subsection{Scale variation in evolution of PDFs}
\subsection{Varying two scales together}
\subsubsection{Electroproduction}
\subsection{Hadroproduction}

\subsection{Scale variation for two processes}

\section{Building the theory covariance matrix}
\label{sec:prescrip}
\subsection{Symmetric prescriptions}
\subsection{Asymmetric prescriptions}
\subsection{Categorising processes}
\subsection{NLO theory covariance matrices}

\section{Validating the theory covariance matrix}
\label{sec:valid}

\section{PDFs with missing higher order uncertainties}
\label{sec:pdfs}

\section{Impact on phenomenology}
\label{sec:mhoupheno}
