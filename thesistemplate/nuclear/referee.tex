\documentclass[a4paper,11pt]{article}
\usepackage[utf8]{inputenc}
\usepackage[a4paper,top=2.5cm,bottom=2.5cm,left=1.5cm,right=1.5cm]{geometry}

\begin{document}

\noindent We thank the referee for the comments on our manuscript,
which we have addressed as follows. In order to facilitate a further revision
we enclose, together with an amended version of the manuscript, also an
annotated version of it, where additions to (deletions from) the original text
are highlighted in blue (red).

\begin{itemize}

\item {\it Somewhere in the introduction it would be ideal to see some (very
  brief) summary of previous studies of the effect of deuteron corrections.
  I am aware that these did occur for both NNPDF2.3 (in a follow-up article)
  and for NNPDF3.0 (in the main article itself). This would help put the
  present approach in context.}

  We have added a new paragraph on Page~2 with a discussion that explains
  what was done in previous NNPDF analyses, as suggested by the referee.

\item {\it I would be interested to know if any experimental correlations are
  normally considered in the NNPDF fit between the proton and deuteron data,
  and if so, how this has been accounted for in the current study? I am not
  requiring, or even suggesting any corrections regarding this. It appears as
  though there is some limited correlation which might be applied, but the
  relevant experimental studies (BCDMS and SLAC) are very old, and some of
  this correlation information seems to appear in references given
  simply as ``private communications''.}

  Experimental correlations between the proton and deuteron data have been
  customarily included in NNPDF analyses for a long time. For BCDMS, the
  systematic uncertainties coming from the calibration of the incoming
  and outgoing muon energy (respectively for the beam and the spectrometer
  magnetic field), from the spectrometer resolution, and from absolute and
  relative normalisations, are fully correlated for all targets and for all
  beam energies, see Sect.~2.1.2 of [JHEP 05 (2002) 062]. For SLAC, the
  absolute and relative normalisation uncertainties are likewise fully
  correlated, for reasons similar to those explained in the same reference.
  In this respect our current analysis is equivalent to NNPDF3.1, and takes
  into account all these correlations whenever the proton and deuteron data
  sets are included in the fit of proton PDFs at the same time. We have added a
  sentence in which we explicitly mention this point in the first paragraph on
  Page~7.

\item {\it In Figure 6 the authors admit that comparing NLO to NNLO PDFs may be
  part of the reason between the discrepancy between the current PDFs and those
  in nNNPDF2.0. Some reference might be given here to NLO/NNLO PDF differences
  previously observed, since these differences are very systematic, depending
  on e.g. NNLO corrections to structure function coefficient functions.}

  Available determinations of nuclear PDFs accurate to NNLO currently include
  only inclusive deep-inelastic scattering measurements. A systematic
  assessment of NLO/NNLO PDF differences was accounted for in two such recent
  analyses: [Phys.Rev. {\bf D100} (2019) 096015] and [Eur.Phys.J. {\bf C79}
    (2019) 471]. We have referenced these analyses in the second 
  paragraph on Page~9. We have also noted that, however, the large PDF
  uncertainties found there (as a consequence of the more limited
  data set than that used in nNNPDF2.0) obscure the phenomenological
  impact of higher-order corrections. 

\item {\it  Since the global-ite2-sh fit contains the shifts of the deuteron
  theoretical predictions I would naively assume it should give a better fit
  quality than ite-2-dw. In fact, the fit in Table~3 is slightly worse. It would
  be good to see some comment on this.}

  The global $\chi^2$ per data point is 1.16 in both the global-ite2-dw and the
  global-ite2-sh fits, as we say in the text (see text on Page~10).
  We inadvertently typeset the wrong number in the last row of Table~3, which we
  have now corrected.

\item {\it It seems to me that given there is no reason to expect the central
  value of the deuteron correction factor to be exactly equal to 1, and indeed
  in Figure~7 it is not, it should unambiguously be the case that ite2-sh is
  the preferred option compared to ite2-dw. It is stated that it is preferred
  on page 10. However, this seems almost an aside, and it is not really
  stressed that this is so in the summary. Unless I have misunderstood
  something I would suggest the conclusion regarding the preferred option
  is presented more clearly.}

  The approach in which nuclear effects give a correction
  with an uncertainty should be preferred to the more conservative one in
  which they give only a larger uncertainty because the uncertainty in the
  nuclear PDFs is correctly estimated by our procedure, and generally smaller
  than the corresponding nuclear correction. This point is introduced in the
  second paragraph on Page~1, and is then discussed in the second paragraph
  on Page~12. We have added a sentence to the concluding paragraph to
  further clarify the preferred option, as suggested by the referee.
  
\end{itemize}  

\noindent We hope that the revised version of our manuscript is now deemed
suitable for publication in Eur.Phys.J.~C.

\end{document}
