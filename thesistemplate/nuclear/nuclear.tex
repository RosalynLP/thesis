\chapter{Nuclear Uncertainties}
The theory covariance formalism developed in Chapter~\ref{chapter:thuncs} can be applied to any source of theory uncertainty in PDFs. One of the most important of these is nuclear uncertainties. A wide range of data is needed to pin down the form of PDFs, including that where the proton is not in a free state. More precisely, this encompasses DIS and DY fixed target measurements involving deuteron and heavy nuclear targets. In these cases the proton's interaction is altered due to the surrounding nuclear environment, and this difference propagates through to the fitted PDFs, leading to an unwanted shift in their central values and uncertainties. We cannot simply discard these data, as they play a crucial role in the strangeness content of the proton and also the light flavour separation at high $x$, a region important for searches for physics beyond the Standard Model. Instead, we must determine corrections to the PDF central value and additional uncertainties to account for the use of nuclear data.

Given their importance, there have been wide-ranging studies of deuteron and heavy nuclear corrections: deuteron corrections have been included in previous PDF determinations via nuclear smearing functions~\cite{Owens:2012bv,Ball:2013gsa,Harland-Lang:2014zoa,Accardi:2016qay,
  Alekhin:2017fpf} based on models of the deuteron wavefunction~\cite{Wiringa:1994wb,Melnitchouk:1994rv,Melnitchouk:1996vp,
  Machleidt:2000ge,Gross:2014wqa}; heavy nuclear corrections have been included following a selection of nuclear models~\cite{Harland-Lang:2014zoa,Dulat:2015mca,Alekhin:2017kpj} or fitting the data~\cite{Accardi:2016qay}. Using such models, however, can introduce a bias that is difficult to quantify. In the past, NNPDF has opted to ignore nuclear effects on the assumption that they are small~\cite{Ball:2013gsa, Ball:2014uwa, Ball:2017nwa}, however this is another source of uncertainty that is becoming increasingly important as PDF precision increases. Furthermore it is thought that the shape of PDFs can be affected, especially at high $x$~\cite{Owens:2012bv}, and this was evidenced in previous NNPDF fits with deuteron corrections following Eq.~(8) of~\cite{Harland-Lang:2014zoa}, with parameter values
from~\cite{Martin:2012da}; however, an increase in $\chi^2$ here suggested that the nuclear uncertainty was not effectively determined.
  
In this Chapter we show how to account for nuclear effects, both deuteron and heavy nuclear, in proton PDF fits. We do this using a theory covariance matrix of nuclear uncertainties, and propose two alternatives for their inclusion: one is to simply apply a nuclear uncertainty, effectively deweighting the affected datasets in the PDF fit proportionally; the other is to shift the PDF central values by applying a nuclear correction, applying smaller nuclear uncertainties as a result. If the shift is estimated accurately, then for an uncertainty smaller than the shift the second method gives a more precise outcome.

We can determine nuclear corrections by comparing the theory predictions for nuclear observables using proton PDFs with those using the correct nuclear PDF (nPDF). This shift can be identified with Eqn~\ref{eqn:thshift} in Chapter~\ref{chapter:thuncs}, i.e. quantifying the size of nuclear correction for that data point. The collective shifts can then be used to construct a theory covariance matrix based on Eqn~\ref{eqn:covmat_format_def}. In carrying out this work we looked first at heavy nuclear corrections (for Cu, Fe and Pb) and then at deuteron corrections, addressing them separately because deuterons, being only a proton and a neutron, are distinct from a heavy nuclear environment such as $^{56}$Fe, with 26 protons and 30 neutrons bound together. 

For the heavy nuclear PDFs we initially used~\cite{Ball:2018twp} a combination of three available nPDF sets (DSSZ~\cite{deFlorian:2011fp},
nCTEQ15~\cite{Kovarik:2015cma}, and EPPS16~\cite{Eskola:2016oht}), but NNPDF subsequently released its own global nPDFs, nNNPDF2.0~\cite{AbdulKhalek:2020yuc}, which is what we will consider in this Chapter. Given the enhanced difficulty of nPDF determination, all of these nPDF sets are only available at NLO. For the deuteron PDFs we developed a self-consistent iterative procedure to determine deuteron PDFs at NNLO within the NNPDF formalism, and used the output of this to determine deuteron corrections. These deuteron PDFs have the advantage over those from nNNPDF2.0 that they are NNLO, but are based on less data so have larger uncertainties. This should at worst lead to a conservative uncertainty estimation, but we will discuss the comparison in Section~\ref{sec:summandoutlook}.

This chapter is organised as follows. First we review the nuclear data in proton PDF fits (Sec.~\ref{sec:nucdat}). Then we consider heavy nuclear uncertainties including the resulting covariance matrix and the PDFs including these corrections. We then look at deuteron uncertainties in the same way (Sec.~\ref{sec:deutunc}), before summarising the results in Sec.~\ref{sec:summandoutlook}.

\section{Nuclear data in PDFs}
\label{sec:nucdat}
We consider the NNPDF3.1 NNLO dataset which consists of $\sim$ 4000 data points, of which $\sim$ 10\% is deuteron data and $\sim$ 20\% is heavy nuclear data. The table below summarises the datasets which make up the total nuclear data, giving the name of dataset, the observable it corresponds to, and the nuclear target involved.
\begin{center}
%\rowcolors{3}{blue!40!}{blue!20!}
\begin{tabular}{ |p{4cm}|p{7cm}|p{2cm}|  }
\hline
\multicolumn{3}{|c|}{Nuclear data} \\
\hline
Dataset & Observable & Target \\
\hline
\rowcolor{blue!20}
SLAC~\cite{Whitlow:1991uw}& DIS structure functions $F_2^d$ & Deuterium  \\
\rowcolor{blue!20}
BCDMS~\cite{Benvenuti:1989fm} & DIS structure functions $F_2^d$ & Deuterium \\
\rowcolor{blue!20}
NMC~\cite{Arneodo:1996kd} & DIS structure function ratios $F_2^d/F_2^p$ & Deuterium \\
\rowcolor{blue!20}
DYE866/NuSea~\cite{Towell:2001nh} & DY cross section ratios $\sigma^{\rm DY}_{pd}/\sigma^{\rm DY}_{pp}$ & Deuterium \\
\hline
\rowcolor{blue!40}
CHORUS~\cite{Onengut:2005kv} & CC DIS cross sections & $^{208}_{82}$Pb  \\
\rowcolor{blue!40}
NuTeV~\cite{Tzanov:2005kr} & DIS dimuon cross sections & $^{56}_{26}$Fe  \\
\rowcolor{blue!40}
DYE605~\cite{Heinrich:1989cp} & DY dimuon cross sections & $^{64}_{32}$Cu  \\
\hline
\end{tabular}
\end{center}
\section{Heavy nuclear uncertainties}
\label{sec:hnucunc}
Heavy nuclear uncertainties can be included using the covariance matrix methodology discussed in Chapter~\ref{chapter:thuncs}.
Each contribution to the covariance matrix, can be determined
\be 
\Delta_i^{(k)} = T_i^N[f_N^{(k)}] - T_i^N[f_p],
\ee 
where $T_i^N$ are the nuclear observables for the $i$-th data point, $f_N^{(k)}$ is a nuclear PDF replica indexed by $k$ and $f_p$ is the central value of the proton PDF. This is the difference between the obervable calculated with a given replica of the ``correct" nuclear PDF and the current value, which is calculated with the proton PDF. Including $N_{rep}$ contributions, one for each replica, $k$, means that the uncertainty in the nPDF is automatically incorporated into the nuclear uncertainty. A covariance matrix can then be constructed as
\be 
S_{ij} = \frac{1}{N_{rep}} \sum_k^{N_{rep}} \Delta_i^{(k)} \Delta_j^{(k)} .
\ee
A more ambitious approach is to also try and correct the value of the nuclear observable we use, so that it is based on the nPDF rather than the proton one. This can be done by applying a shift,
\be 
\delta T_i^N = T_i^N[f_N] - T_i^N [f_p],
\ee
to the nuclear observables. In this case we must amend the contributions to the covariance matrix so that they are
\be 
\Delta_i^{(k)} = T_i^N[f_N^{(k)}] - T_i^N[f_N].
\ee 
If the shift is accurate enough and greater than the uncertainty then this latter method should lead to a better fit. 

In the above equations the nuclear observables, $T_i^N$, are calculated from the proton observables, $T_i$, by taking into account the non-isoscalarity of the target, i.e. by combining the proton and neutron observables in accordance with the atomic number, $Z$ and mass number, $A$. Explicitly,
\be 
\begin{split}
T_i^N[f_p] &= \frac{1}{A} \bigg( Z T_i[f_p] + (A-Z) T_i[f_n] \bigg), \\
T_i^N[f_N] &= \frac{1}{A} \bigg( Z T_i[f_{p/N}] + (A-Z) T_i[f_{n/N}] \bigg).
\end{split}
\ee
The first line is what is done in standard NNPDF fits, and the second line is the extension to the nPDF case. Here $f_{p/N}$ is the PDF for the proton bound in a nucleus, $N$, and $f_{n/N}$ is the same for the neutron. We assume that the two are related by swapping $u$ and $d$ quarks. We obtain these PDFs directly from nNNPDF2.0, but they relate to $f_N$ via
\be 
f_N = \frac{1}{A} \bigg( Z f_{p/N} + (A-Z) f_{n/N} \bigg).
\ee
%%%%%%%%%%%%%%%%%%%%%%%%%%%%%%%%%%%%%%%%%%%%%%%%%%%%%%%%%%%%%%%%%%%%%
\begin{figure}[h]
  \begin{center}
    \includegraphics[width=\linewidth]{nuclear/plots/observable_ratio_nuclear.png}
   \caption{ Ratio between the nuclear observables computed with nPDFs, $T_i^N[f_N]$, and the central prediction computed with proton PDFs, $\langle T_i^N[f_p] \rangle$. The error is the standard deviation of the distribution of $T_i^N[f_N]$ replicas. Data are organised in bins of increasing $(x, Q^2)$ within each dataset. 
    \label{fig:nucobs} }
  \end{center}
\end{figure}
%%%%%%%%%%%%%%%%%%%%%%%%%%%%%%%%%%%%%%%%%%%%%%%%%%%%%%%%%%%%%%%%%%%%%%

Before proceeding to the covariance matrix itself, we can first investigate the change to the nuclear observable that arises from using nPDFs rather than proton ones. Fig.~\ref{fig:nucobs} shows the ratio between these two values for the heavy nuclear datasets. We can see in all datasets that there is a kinematic dependence, although this is especially evident in CHORUS. This is a result of the kinematic dependence of the ratio of proton and nuclear PDFs. CHORUS $\nu$ and NuTeV $\nu$ data in particular show a systematic shift downwards which is not comfortably within errors. This suggests that applying a shift as well as an uncertainty could be a sensible strategy. 


\subsection{The heavy nuclear covariance matrix}
We now turn to the covariance matrix for heavy nuclear uncertainties. Fig.~\ref{fig:nuccov} shows (top panel) the square roots of the diagonal elements of this covariance matrix, which are equivalent to the \% per-point uncertainties. The plot is for the deweighted case, i.e. where a shift is not applied, but rather the datasets are just deweighted by a nuclear uncertainty. However, the general pattern does not change in the covariance matrix for the shifted case. 

It is clear that the heavy nuclear uncertainties are comparable to the experimental uncertainties and are larger in most regions other than CHORUS $\bar{nu}$. This suggests that all datasets apart from that will be significantly deweighted in the fit. We see that the plot has many features in common with Fig.~\ref{fig:nucobs}, in particular the kinematic pattern, and this makes sense as the covariance matrix is composed using the difference in observables when using nPDFs versus proton ones. 
%%%%%%%%%%%%%%%%%%%%%%%%%%%%%%%%%%%%%%%%%%%%%%%%%%%%%%%%%%%%%%%%%%%%%
\begin{figure}[H]
  \begin{center}
    \includegraphics[width=\linewidth, trim={4cm 0 4cm 0}]{nuclear/plots/diag_covmat_deweighted.png}
    \includegraphics[width=0.45\linewidth]{nuclear/plots/covmats_Experiment_nuclear.png}
    \includegraphics[width=0.45\linewidth]{nuclear/plots/covmats_Total_nuclear.png}
    \includegraphics[width=0.45\linewidth]{nuclear/plots/covmats_Total_shifted_nuclear.png}
   \caption{ \textbf{Top panel:} Square root of diagonal elements of covariance matrices for $C$ (purple), $S$ (orange) and $C+S$ (blue). All values are displayed as a \% of data. \textbf{Lower panels}: Correlation matrices for heavy nuclear data. The experiment correlation matrix, $C$, is shown alongside the total correlation matrix, $C+S$, for both the deweighted and the shifted case.
    \label{fig:nuccov} }
  \end{center}
\end{figure}
%%%%%%%%%%%%%%%%%%%%%%%%%%%%%%%%%%%%%%%%%%%%%%%%%%%%%%%%%%%%%%%%%%%%%%

The lower panels of Fig.~\ref{fig:nucobs} investigate the pattern of correlations, displaying correlation matrices as defined in Sec.~\ref{sec:results} of Chapter~\ref{chapter:mhous}.
\subsection{PDFs with heavy nuclear corrections}
\section{Deuteron uncertainties}
\label{sec:deutunc}
\subsection{The deuteron covariance matrix}
\subsection{PDFs with deuteron corrections}
\section{Summary and outlook}
\label{sec:summandoutlook}
