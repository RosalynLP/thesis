\chapter{Abstract}

We are now in the era of high precision particle physics, spurred on by a wealth of new data from the Large Hadron Collider (LHC). In order to match the precision set by modern experiments and test the limits of the Standard Model, we must increase the sophistication of our theoretical predictions. Much of the data available involve the interaction of protons, which are composite particles. These interactions are described by combining perturbative Quantum Chromodynamics (QCD) with parton distribution functions (PDFs), which encapsulate the non-perturbative behaviour. Increasing accuracy and precision of these PDFs is therefore of great value to modern particle physics.

PDFs are determined by multi-dimensional fits of experimental data to theoretical predictions from QCD. Uncertainties in PDFs arise from those in the experimental data and theoretical predictions, as well as from the methodology of the fit. At the current levels of precision theoretical uncertainties are increasingly significant, but have so far not been included in PDF fits. Such uncertainties arise from many sources, an important one being the truncation of the perturbative expansion for the theoretical predictions to a fixed order, resulting in missing higher order uncertainties (MHOUs).

In this thesis we consider how to include theory uncertainties in PDF fits by constructing a theory covariance matrix and adding this to the experimental one. We then use this method to address several sources of uncertainties. MHOUs are estimated and included as a proof of concept in next-to-leading order (NLO) PDFs. We find that these capture many of the important features of the known PDFs at the next order above (NNLO). We then go on to investigate uncertainties from nuclear effects, estimate their magnitude and assess the impact of their inclusion on the PDFs. Finally, we consider the common use of making predictions using PDFs, for the case where PDFs also include theoretical uncertainties. Here there can be correlations between the PDFs and the predictions, which can lead to a shift in the predictions and their uncertainties. We show these effects to be generally small, and provide formulae for their exact evaluation.
