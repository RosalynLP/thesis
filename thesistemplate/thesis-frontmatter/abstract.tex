\chapter{Abstract}

We are now in the era of high precision particle physics, spurred on by a wealth of new data from the Large Hadron Collider (LHC). In order to match the precision set by modern experiments and test the limits of the Standard Model, we must increase the sophistication of our theoretical predictions. Much of the data available involves the interaction of protons, which are composite particles. Their interactions are described by combining perturbative Quantum Field Theory (QFT) with parton distribution functions (PDFs), which encapsulate the non-perturbative behvaiour. Increasing accuracy and precision of these PDFs is therefore of great value to modern particle physics.

PDFs are determined by multi-dimensional fits of experimental data to theoretical predictions from QFT. Uncertainties in PDFs arise from those in the experimental data and theoretical predictions, as well as from the methodology of the fit. At the current levels of precision theoretical uncertainties are increasingly significant, but have so far not been included in PDF fits. Such uncertainties arise from many sources, an important one being the truncation of the perturbative expansion for the theoretical predictions to a fixed order, resulting in missing higher order uncertainties (MHOUs). 

In this thesis we consider how to include theory uncertainties in future PDF fits, and address several sources of uncertainties. MHOUs are estimated and included as a proof of concept in next-to-leading order (NLO) PDFs. We find that these capture many of the important features of the known PDFs at the next order above (NNLO). We then go on to investigate uncertainties from previously ignored heavy nuclear effects and higher twist effects, estimate their magnitude and assess the impact of their inclusion on the PDFs.